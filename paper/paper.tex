% Options for packages loaded elsewhere
\PassOptionsToPackage{unicode}{hyperref}
\PassOptionsToPackage{hyphens}{url}
%
\documentclass[
]{article}
\usepackage{lmodern}
\usepackage{amssymb,amsmath}
\usepackage{ifxetex,ifluatex}
\ifnum 0\ifxetex 1\fi\ifluatex 1\fi=0 % if pdftex
  \usepackage[T1]{fontenc}
  \usepackage[utf8]{inputenc}
  \usepackage{textcomp} % provide euro and other symbols
\else % if luatex or xetex
  \usepackage{unicode-math}
  \defaultfontfeatures{Scale=MatchLowercase}
  \defaultfontfeatures[\rmfamily]{Ligatures=TeX,Scale=1}
\fi
% Use upquote if available, for straight quotes in verbatim environments
\IfFileExists{upquote.sty}{\usepackage{upquote}}{}
\IfFileExists{microtype.sty}{% use microtype if available
  \usepackage[]{microtype}
  \UseMicrotypeSet[protrusion]{basicmath} % disable protrusion for tt fonts
}{}
\makeatletter
\@ifundefined{KOMAClassName}{% if non-KOMA class
  \IfFileExists{parskip.sty}{%
    \usepackage{parskip}
  }{% else
    \setlength{\parindent}{0pt}
    \setlength{\parskip}{6pt plus 2pt minus 1pt}}
}{% if KOMA class
  \KOMAoptions{parskip=half}}
\makeatother
\usepackage{xcolor}
\IfFileExists{xurl.sty}{\usepackage{xurl}}{} % add URL line breaks if available
\IfFileExists{bookmark.sty}{\usepackage{bookmark}}{\usepackage{hyperref}}
\hypersetup{
  hidelinks,
  pdfcreator={LaTeX via pandoc}}
\urlstyle{same} % disable monospaced font for URLs
\setlength{\emergencystretch}{3em} % prevent overfull lines
\providecommand{\tightlist}{%
  \setlength{\itemsep}{0pt}\setlength{\parskip}{0pt}}
\setcounter{secnumdepth}{-\maxdimen} % remove section numbering

\author{}
\date{}

\begin{document}

\hypertarget{header-n0}{%
\subsection{Introduction}\label{header-n0}}

In 1975, the year that Margaret Thatcher won the Conservative leadership
election, Anthony King, then a political scientist at the University of
Essex, published a paper titled \emph{Overload: Problems of Governing in
the 1970s} \#kingOverloadProblemsGoverning1975 . King argued that
governing had become nearly impossible. The pressure of elections forced
governments to abandon long-term thinking. To survive, governments had
no other option than to concede to electoral and trade union demands,
hereby effectively functioning as an "unlimited insurance company".
Consequently, King argued, the public expenditure and the public sector
had expanded dramatically, a process that evolved against the backdrop
of the post-1973 economy of stagflation and rising unemployment.
Combined with technological innovation and administrative incompetence
this had produced a vicious circle of ungovernability and "overload".
King captured this "overload thesis" in the metaphor of Goethe's
\emph{Zauberlehrling}. After giving away control to an enchanted
broomstick, the sorcerer's apprentice cannot go back. The waters rise.
There is nothing to be done, except for waiting for the sorcerer to
return.

"Overload" and the diagnosis of "ungovernability" became crucial
elements in the paradigm shift that took place in the late 1970s. In
this period, keynesian recipes to crisis and the paradigm of socialist
planning crumbled. During the Wilson and Callaghan government, a period
of "Knightean uncertainty" appeared
\#blythGreatTransformationsEconomic2002 . The overload thesis can be
seen as the diagnosis that produced such uncertainty. It pointed at the
problems of big government, overspending and electoral pressures, hereby
paving the sorcerer's return in the form of Margaret Thatcher's victory
in 1979.

Current literature emphasizes the constructed nature of the various
'crises' that befell the 1970s Labour governments and the Keynesian
policy paradigm. In his seminal article on the topic Colin Hay, for
example, shows how the 'Winter of Discontent' - the winter of 1978/79
that saw widespread strikes and power outages - was both a political
reality as a medial construct effectively exploited by Conservative
strategists and ideologues \#hayNarratingCrisisDiscursive1996
\#hayRethinkingCrisisNarratives1995 . Hay's premise can be recognized in
the field of constructivist institutionalism, that sees periods of
economic crisis such as the late 1970s as times of 'Knightean
uncertainty', when actors are not only uncertain about the solutions to
crisis, but also about their interests in it
blythGreatTransformationsEconomic2002 \#hay2004ideas .

Consequentially, the intellectual history of the Conservative victory in
1979 is often written as the history of a "revolution" or "moment". A
series of simmering crises culminated in the "winter of discontent'' and
the sudden defeat of the Callaghan government. A substantial number of
contributions have question this temporality of the neoliberal revolt
\#hiltonNewTimesRevisited2017 . Most of them focus on the existence of
neoliberal ideas long before Thatcher and the policy proposal flowing
from think tanks such as the Atlas Institute and the Institute for
Economic Affairs \#jacksonCurrentsNeoLiberalismBritish2016 . The
chronology of the construction of political crisis, however, is still
focussed on the late 1970s. This perspective suffers from two
shortcomings. First, its temporality of "moments'' leaves no room for
longer-term ideational change. Second, it renders the agency the actors
associated with the "old'' ideas impossible. The advocates of the
crumbling Keynesian paradigm can only stand by and watch how media
discursively construct crisis and Conservative ideas make use of the
uncertainty. Lastly, the approach also risks indirectly perpetuating the
story told by the "neoliberals'' themselves. By presenting the years
1978-79 as a "Conservative Opportunity'', one risks narrating the same
story necessity as the Conservatives themselves did
\#blakeConservativeOpportunity1976.

This paper complements the literature on the New Right "moment" in
1978-79 by showing how another crisis was discursively constructed: the
crisis of bureaucracy. This crisis concerned the argument(s) that the
processes of nationalisation, centralisation and trade union concessions
gave rise to overtly large, costly, inefficient and unaccountable
bureaucracy. Excessive bureaucracy was one of the demons that haunted an
overloaded government. However, the issue of bureaucracy is noteworthy
for two reasons. First, ideas about maladministration, unaccountable
bureaucrats and an costly civil service are by no means tied to the
topic of overload or the mid-1970s. Second, ideas about the state and
the civil service play a marginal role in the historiography of the
neoliberal turn. They are generally assumed to be secondary to economic
ideas about planning.

\hypertarget{header-n7}{%
\subsection{Problem Statement and Question}\label{header-n7}}

Thus a puzzle appears: how do ideas about the shape and character of
bureaucracy relate to the intellectual shift in the late 1970s "overload
thesis'' in particular if these ideas are not unique to the moment of
(discursively constructed) crisis? By studying arguments pertaining to
bureaucracy, this paper investigates the 'discursive construction of
crisis' in order to better understand the long(er)-term history of such
a construction.

\hypertarget{header-n9}{%
\subsection{Approach and Method}\label{header-n9}}

This paper studies the construction of bureaucratic crisis by
conceptualizing an argumentative apparatus pertaining to bureaucracy.
The paper argues that the "overload thesis'' was embedded in
argumentative changes that long predated the mid-seventies. These
changes were rooted in several contextual factors, such as the Labour
emphasis on "efficiency'' and "participation'' in respectively the late
sixties and early seventies and the debates on the EEC accession, the
Local Government Act and devolution in the early seventies.

The argumentative changes that are discussed in this paper include:

\begin{itemize}
\item
  a substantialization of arguments: bureaucracy is less about "distant"
  communism and less used in a figurative sense. Instead, it becomes
  something tangible, related to concrete problems in the administration
\item
  a temporalisation of arguments and the increasing: bureaucracy becomes
  entangled in a temporal frame of "unstoppable growth", but also in a
  temporal narrative of the "keynesian" 1970s and the need for something
  else in the next decade.
\item
  an interrelatedness of arguments: arguments about cost, efficiency,
  size and accountability are increasingly combined in a comprehensive
  critique of overloaded government.
\end{itemize}

The paper combines a close reading of parliamentary proceedings, media
sources and documents produced by party-affiliated persons and
institutions (see next subsection) with a quantitative analysis of the
parliamentary proceedings:

\begin{itemize}
\item
  frequency measures (rel. frequencies through time, collocations
  measures)
\item
  distributional models (word2vec, semantic fields)
\item
  text classification (argument classifier)
\end{itemize}

\hypertarget{header-n27}{%
\subsection{Data}\label{header-n27}}

The paper utilizes three types of data. It concentrates on parliamentary
proceedings as a way to quantitatively detect changes in the use of
arguments. Second, it uses publications by important party institutions,
such as the Fabian Society, the Institute for Economic Affairs and the
Conservative Political Centre.

\hypertarget{header-n30}{%
\subsection{Argumentative Alignment}\label{header-n30}}

\hypertarget{header-n31}{%
\subsubsection{I - Communism, Grievances and Anti-Bureaucratic Rhetoric
(1957-1964)}\label{header-n31}}

In the years of the Conservative MacMillan and Douglas-Home governments,
the different types of arguments identified earlier all surfaced in
parliamentary debates. Bureaucracy was seen as costly, irrational,
inefficient, overly powerful and unaccountable. However, three features
mark the use of the concept in the period between 1957 and 1964: the
extent to which bureaucracy was linked to Soviet socialism by
Conservative actors; the overall use of the concept as a rhetorical
device in arguments not targeted at bureaucracy itself and the focus on
a bureaucratic "mentality" as the object of polemic.

First, the majority of speeches using the concept of "bureaucracy" were
not discussing the concept itself. This "acentrality'' of the concept
resulted in the use of bureaucracy as a form of figurative speech, or
the use of the concept in arguments that were not targeted at
bureaucracy itself. Indicative is also the ratio between the frequencies
of "bureaucracy'' and "bureaucratic''. Before the mid-1960s, this ratio
was moving around 1, meaning that the noun and the adjective had
comparable frequencies. Later, the ratio increases, meaning that
bureaucracy was increasingly used when compared to "bureaucratic''.

IMAGE: ADJECTIVE-NOUN RATIO

When bureaucracy itself was the subject of debate, it were mostly
Conservative MPs that fulminated against the bureaucratic consequences
of socialist policies. Heathcoat Amory, Conservative MP and Chancellor
of the Exchequer, argued in a 1959 speech that the supreme issue in the
next election would be "a choice between freedom and opportunity for the
individual and the all-pervading power of a bureaucratic state''
\#BureaucracyFreedom1959 . For the Conservatives, bureaucracy was the
simple byproduct of socialism, a sentiment only reinforced by the Cold
War tensions and not restricted to Conservatives alone. Labour critiques
on the danger of Soviet-inspired bureaucratic totalitarianism were part
of broader debates in socialist circles in the late 1950s. They were an
integral part of the clashes between revisionists (Gaitskellites, named
after the Labour leader Hugh Gaitskell) and orthodox socialists
("Bevanites"). The latter collective was accused by the centrist segment
of the party of supporting a bureaucratic dictatorship. However, the
concept of bureaucracy was also used by the left wing factions of the
party as a way to critique the corrupting managerial class that existed
under capitalism. In the late 1950s a third intellectual collective
formed that also took bureaucracy as its nemesis. With the Soviet Union
generally discredited after the suppression of the 1956 Hungarian
uprising scholars such as Stuart Hall, Raphael Samuel and Charles Taylor
started formulating ideas that would later become part of the "New
Left''. Based on amongst other C. Wright Mills' \emph{Power Elite}
(1956) they formulated more substantive critiques of bureaucratic power,
technocracy and "complex power structure of the giant oligopolies, in
which wealth, power, status and control are interchangeable and
interlinked" \#davisArguingAffluenceNew2012,
\#gearyBecomingInternationalAgain2008 .
\#gearyBecomingInternationalAgain2008. The concept of a "power elite'' -
as well as other theories on bureaucratization and managerialism - was
used to draw the attention to the de facto capitalist power of
government bureaucrats and the bureaucratic power of capitalist
corporations. As such, bureaucracy became a more central concept in
attacks on both senseless planning and capitalist big business
\#foksSociologicalImaginationBritish2018. The emphasis of bureaucracy as
a systemic feature of communism, or bureaucracy as a harmful effect of
capitalism is reflected in the central terms as aggregated by a language
model trained on the period 1958-1961. Terms referring to the Soviet
spectre, such as "dictatorship" and "communism" appear as central in the
semantic field around bureaucracy.

Besides these ideology-driven suspicions of bureaucratic power,
day-to-day parliamentary rhetoric also featured anti-bureaucratic
arguments. With Labour in opposition until 1964, it were Labour MPs that
most frequently employed these arguments. What marked complaints about
bureaucracy in the late 1950s and early 1960s is the emphasis on
arguments pertaining to irrationality and inefficiency. Labour MPs, but
also Conservative members, frequently lamented a bureaucratic
"mentality": slow, inefficient administrative "bumbledom", captured in
idioms such as "formfilling", "red tape", "paperwork" and phrases such
as "bureaucracy gone mad" and "bureaucracy gone riot". These allegations
mostly concerned the behavior of specific institutions and the
experiences of constituents passed on by the MPs. Alternatively, this
image of bureaucratic "bogge" was invoked when discussing the potential
effects of legislation. Proposals were rejected because they would lead
to "irresponsible State bureaucracy" (MacColl, Labour, 1958-12-01) or
"short-sighted, reactionary, and restrictive bureaucracy" (Noel-Baker,
Labour, 1961-11-15). This everyday use of the concept as a way to attack
the injustices caused by petty bureaucrats shows from terms such as
"bribery", "corruption" and "disrepute". Furthermore, an aggregation of
adjectives referring to "bureaucracy" also shows the importance of
clusters consisting of "unnecessary", "excessive", "petty" and
"monstrous".

IMAGE: MOST CENTRAL TERMS

IMAGE: SEMANTIC FIELD

IMAGE: ADJECTIVES

\hypertarget{header-n40}{%
\subsubsection{II - Efficiency and Temporalisation
1964-1970}\label{header-n40}}

In 1964 Harold Wilson won the election after campaign rhetoric focused
on modernisation, reform and harnessing the "white heat of technological
revolution" to bring Britain up to speed. His revolution also entailed a
different approach to government and the civil service focused on
"efficiency" and "modernisation". Wilson was a "Centre Left Technocrat",
a collective that overlapped to a large extent with the Fabian Society.
In the 1964 tract \emph{The Administrators} (1964) this movement of
pragmatist socialists had set an agenda for civil service reform,
including the shift to specialist bureaucrats (instead of the
Oxbridge-schooled humanists), a more open government and more
accountability. Labour's agenda of efficiency was purely pragmatic. It
differed from the earlier planning adepts, that put forward planning's
ethical dimension. Instead, planning was meant to ensure growth and
productivity \#favrettoWilsonismReconsideredLabour2000. Besides these
political goals, the insistence on bureaucratic efficiency and effective
statecraft also stemmed from more complex factors. David Edgerton notes
how since the ensuing of the "two cultures" debate in 1956, the British
civil service was increasingly accused of being composed of uninformed
generalists. This also involved a generational conflict, since a new
educated "technocratic" middle-class, also present in the Fabian
Society, rose to power edgertonSnowAntiHistorianBritish2005.

Historians consider the practical outcomes of Wilson's agenda rather
marginal. Perhaps the most influential outcome was the Fulton Report,
published in 1968 after a two-year long enquiry into the functioning of
the civil service. The report proposed a host of reforms, of which
little were realised. Nevertheless, the late 1960s saw some
institutional reconfigurations, as well as the appearance of explicit
discussions about the nature and functioning of bureaucracy.

The institutional effects of Wilson's agenda and the Fulton Committee
are considered to be marginal. The argumentative apparatus, however,
changed during the years of Labour government. A first observation is
the plain rise in frequency of the word "bureaucracy". Especially
Conservatives uttered the term more frequently, something that is not
surprising given their natural antipathy towards socialism and the
associated high level of bureaucracy. However, Conservatives used the
term more than Labour had done during their time in opposition.
Furthermore, Labour itself also used the term more and more after 1968.

Several patterns appear from the visualisations of the different
argument types and their diachronic development. Labour practically
stopped using anti-bureaucratic arguments pertaining to the size of
bureaucracy and its irrational nature. This likely stems from their role
as government party, a role that made the chance of complaints about
overly big or extremely irrational bureaucracy smaller. Surprisingly,
argument relating to accountability also decrease in Labour discourse,
although after 1968 - the year of the Fulton Report - this type
increases in frequency. With regard to the Conservatives, it is striking
how arguments pertaining to efficiency are shared among the parties.

Besides the dynamics \emph{within} the argumentative apparatus, there
also appears a more general form of change: a \textbf{temporalisation}
of the arguments. In speeches by both Labour and Conservative MPs, there
is a growing sense of continuous expansion. In other words, in the late
1960s, bureaucracy is increasingly seen as in the process of expansion.
In 1968, this issue is discussed explicitly when Geoffrey Rippon, who
would later compose the EEC accession bill, filed a motion titled 'the
growth of bureaucracy' on the 29th of January 1968. The motion started
as follows:

\begin{quote}
I beg to move, That this House deplores the continued growth of
bureaucracy and the failure of Her Majesty's Government to announce in
their Statement on Public Expenditure, Command Paper No. 3515, clear
proposals to streamline the machinery of Government and so reduce the
numbers employed in the public service. The most urgent task facing
Parliament today is to curb the power of central Government and close
the ever-widening gap between Whitehall and our constituents. In the
last three years, we have seen an apparently uncontrollable growth in
the numbers of public servants. Whereas between 1951 and 1964 the number
of non-industrial civil servants, excluding the Post Office, fell by
11,000, in the three years between October, 1964, and October, 1967, the
numbers rose by no fewer than 54,000. This is an increase in the hard
core of administrative bureaucracy and represents a rise of about 13 per
cent (Rippon, 29-01-1968)
\end{quote}

The motion, as well as the extensive debate that ensued, is an early
example of a more explicit and conscious Conservative attack on
bureaucracy that involved multiple types of anti-bureaucratic arguments.
Rippon mentions both the "evils'' of inflexibility and arbitrariness as
well as the cost and centralisation.

Labour MPs reacted scornfully and accused Rippon of questioning the
integrity of the civil service. Harold Lever, Financial Secretary to the
Treasury, gave a more substantive answer. The growth in the Civil
Service, according to Lever, "is absolutely inevitable in any modern
society". Bureaucracy grows because of the increase in white-collar
labor, the greater demand put on services by virtue of a more affluent
society and (only) thirdly, the responsibility of government in the
national economic effort which put a greater demand on government. Lever
concluded by saying that "{[}u{]}nless the Tory Party wishes to make
itself the last repository of out-of-date and irrelevant prejudice it
might as well admit that, if we envisage a growing and modernising
society, it automatically follows that there must be an increase in the
public services as well as in private service industries. Any other view
is as stupid as it is doctrinaire" (Lever, 29-01-1968).

What this debate shows is the addition of a temporal frame to the
argumentative apparatus. The problem was not only the size and cost of
bureaucracy, but the historical fact that bureaucracy had grown and the
expectation that it would continue to do so in the future. Signs of this
process of temporalisation also appear from collocation analysis. Words
such as "swollen'', "enormous'' and "growth'' each peak in the Wilson
years. As seen in the debate in 1968, Labour responded to the
allegations of bureaucracy growing out of control by pointing at the
growth of bureaucracy as both a natural phenomenon in advanced
industrial societies and the consequence of an interventionist state.
However, they also adopted the temporality of growth from time to time
by arguing that it was actually the Conservative government before 1964
that was responsible for the growth.

IMAGE: ARGUMENT SHARES PER PARTY\\
IMAGE: SEMANTIC FIELD

\hypertarget{header-n52}{%
\subsubsection{III - Heath, Europe and Increasing Topicality 1970 -
1974/75}\label{header-n52}}

\hypertarget{header-n53}{%
\paragraph{Europe and Devolution}\label{header-n53}}

Political debate in the years of the Conservative Heath government was
marked by the extensive discussion of several issues closely related to
and involved with arguments pertaining to bureaucracy. First, the
accession to the European Community provoked a heightened topicality of
the concept of bureaucracy. In January 1972, the Conservative launched
the European Communities Act. In February 1972 intense debate followed
during the Second Reading. Three years later, the referendum on
membership, pledged by the Labour Party in its October 1974 Manifesto
provoked new debates.

The second issue was devolution or 'Home Rule"': the delegation of
administrative and legislative powers to subcentral levels, most
relevant to Scotland, Wales and Northern Ireland
\#mitchellDevolutionUK2013. With nationalist tensions rising in the late
1960s devolution became a more pressing political issue, and both
parties revisited their long-standing positions on the topic. In 1968 a
Royal Commission was established that investigated the potential for
legislative devolution. A report was published in 1973 and lead to
fierce debates in the Commons. Especially the Conservative Party,
officially named the Conservative and Unionist Party, was not eager to
devolve \#ezzamelAccountingPoliticsDevolution2008 . During the
tumultuous year 1974 the Scottish National Party won eleven seats (in
the October election) and the devolution debate became even more
pressing. In reaction, the newly instated Labour government published a
White Paper in 1975 with further plans for devolution. Many Labour MPs,
however, were fiercely against it.

The debates on the European accession and the issue of devolution
produced a plethora of bureaucratic arguments. The EEC and "Brussels"
was consistently referred to by all sides as a "bureaucracy". Most
frequent were arguments related to accountability, voiced mostly by
Labour MPs. In the case of devolution, the threat to local government
and individual freedom was often mentioned. Questions of cost efficiency
appeared in both contexts. Both topics not only lead to an increased use
of the concepts and its related arguments, but also changed the
argumentative apparatus in multiple ways. First, the debates further
\textbf{\emph{politicised}} the concept of bureaucracy. That is, because
of devolution and the membership issue, the concept of bureaucracy was
drawn into highly politicised debates. Using sentiment lexicons, the
average polarity of a debate can be measured, showing this politicised
nature of mostly the EEC accession debate. Besides the politicization,
the debates also accelerated the integration of different arguments and
different contexts. Constellations of several different argument types
became more familiar. Devolution and the EEC were even discussed
together under the umbrella of bureaucracy. The Labour MP Frank Ashcroft
Judd argued in 1971 that "at a time when so many politicians on all
sides express their concern about the growth of impersonal and remote
bureaucracy in Britain and the need for effective devolution and finding
new ways of democratic participation, it is odd that we should be so
ready to move into an international group which has a still more
powerful bureaucracy over which it will be even more difficult to
exercise control" (Judd, 1971-01-20).

IMAGE: PRODUCTIVITY

Stances towards bureaucracy also changed as the result of ideological
shifts in the Labour Party. The New Left, that launched early attacks on
bureaucratic institutions in the early 1960s, had become an important
voice in the party and in Europe in general. The solution or alternative
offered by the New Left was "participation". In a Fabian Tract published
in 1973, participation is seen as both an end in itself (a "means to
self-realisation") as a way to achieve better decisions and to combat
the power of centralised bureaucracy. Behind the ideals of participation
in democracy in industry shimmers an acknowledgement of the temporal
frame introduced by the Conservatives in the late 1960s. The development
of participation is, according to the Tract a "necessary response to the
growth in scale services". In 1968, Anthony Crosland, then part of the
Wilson government, responded to the New Left by stating that "there is
no substitute to clerks and computers" except for more reform and
efficiency \#croslandSocialismDangerousWorld1968. In the early
seventies, more and more voices in Labour went beyond calls for more
efficiency and developed more fundamental objections against
"bureaucracy".

\hypertarget{header-n59}{%
\subsubsection{IV - Crisis, Callaghan and the Sorcerer's
Return}\label{header-n59}}

In 1974 Wilson returned to power, albeit without a majority. For this
reason, the second election of 1974 was organised in October, resulting
in a slight increase in Labour seats. Only two years later, Wilson was
succeeded by Jim Callaghan. One year later, the party lost its majority,
leading to deal bargaining with several other parties. When a deal with
the SNP collapsed in 1979, the Conservatives lead by Thatcher won the
election in May 1979. The tumultuous period between 1974 and 1979 cannot
be described extensively here. However, a glance at frequency plots
shows how dramatically "bureaucracy" arose as a topic of discussion.
With the growing topicality of bureaucracy driven by the devolution and
EEC debates, the second Wilson government experienced an explosion in
discussions about bureaucracy. Frequencies for the lemma skyrocket
between 1973 and 1977, mostly due to Conservative employment of the
term, but a rising trend is also visible in Labour discourse.

The first year of the new Labour government, 1975, was marked not only
by the publication of King's \emph{overload thesis}, but also by the
rising prominence of Conservative commentary on economic and
administrative policy. The mid-1970s saw the expansion of an
"archipelago of think tanks", consisting of for example the Institute
for Economic Affairs and the Centre for Policy Studies, established by
Thatcher and her mentor Keith Joseph "to do in political terms for the
free market what the IEA had so successfully done in the wider
intellectual community" (Cockett, 1994, 236). The publications by these
institutions quickly adopted King's diagnosis, as visible in Alistair
Burnett's publication "Is Britain Governable?", that was published in
the same year as the institute's establishment.

Outside the think tank's effort to "think the unthinkable", the
Conservative party also increasingly reverted to more ideological
reflections on their course, forced by the electoral defeat and the
image of Edward Heath's frequent "U-turns". Two publications shed light
on this Conservative return to ideological contemplation: the edited
volume \emph{The Conservative Opportunity} (1976) and \emph{The Dilemma
of Democracy} (1978), authored by Quintin Hogg, Lord Chancellor during
the Heath government, and later during the Thatcher administration.

With regard to bureaucracy, the years of the Wilson and Callaghan
government saw the development of a relatively cohesive Conservative
critique on socialist bureaucracy. In their attacks on the Labour Party,
both Blake and Hogg saw bureaucracy as one of the core problems of a
centralised bureaucracy. Blake observes that existing criticisms of
economic policy and the trade union dominance "have been enhanced by -
and closely connected with - a surge of feeling against 'bureaucracy' ".
Hogg, or Lord Hailsam, argues that "At the heart of the elective
dictatorship resides the government machine, the bureaucracy, the Civil
Service". Striking about the "diagnosis" is the moral dimension that is
provided along with the ideological rebuttal of socialism. Similar to
Thatcher, bureaucracy was framed as part of the moral crisis of the
1970s. Bureaucracy was the aesthetic materialisation of uniformity,
greyness, hierarchy and domination \#jacksonMakingThatchersBritain2012a.
Hogg argued that "{[}a{]}ll administration develops into bureaucracy.
Human relations are depersonalized and men begin to think of themselves
as numbers"
\#hailshamofst.maryleboneDilemmaDemocracyDiagnosis1978\_p159. This
aesthetic of bureaucracy was widely felt, and examples are numerous.
Among the most noteworthy is Prince Philip's remark on the issue in a
1977 radio interview. When asked on the expected state of Britain in the
year 2000, the Prince of Wales warned the public for "an increasing
bureaucracy" and "bureaucratic involvement in almost every aspect of the
lives of individual citizens" \#turnerCrisisWhatCrisis2009.

The Conservatives also twisted the socialist expectation of a growing
bureaucracy on its head. Instead of embracing the expansion of
bureaucracy as it was part of a beneficiary expansion of government
responsibilities, Conservatives argued that "{[}i{]}t was tolerable,
even beneficial, in former days precisely because central government
controlled so small a portion of our lives. Now that in almost every
field of activity, government intrudes, it needs to be broken down into
smaller units with divided power" (Hailsam, 165). The \emph{Times}
reported in April 1978 that the attitude of "individualism crossed with
populism" would likely be "in tune with much of the rather vague but
widespread public resentment of bureaucracy and excessive government"
\#ConservativeAgenda1978.

During the years in which the Conservatives developed more substantive
critiques on bureaucracy, Labour MPs and thinkers gradually came to
embrace parts of the Conservative "diagnosis". Writing in a letter to
the Conservative \emph{Times}, Kenneth Younger, retired "Gaitskellite"
Labour MP, wrote in response to one of the CPC pamphlets that "David
Howell's pamphlet on a Tory future is throught-provoking and is
fasionable in its insistence that public spending is ecessive,
bureaucracy inflated and Keynesianism explode"
\#younger.ConservativeViewFuture1976.

On the left flank of the Labour party, attacks were launched against the
unaccountable "mandarins". Michael Williams notes the "common ground
among critics of the civil service on the right of the Conservative
Party and on the left of the Labour Party"
\#williamsCrisisConsensusBritish2000. Brian Sedgemore, for example,
served on the House of Common's Expenditure Committee and its inquiry
into the civil service in 1976-1977. Sedgemore launched polemic attacks
against the unaccountable power held by the civil service. Sedgemore
veheminently attacked the "dangerous role" of the "mandarin clan" and
proposed more efficiency, accountability to both the executive and
Parliament \#DemocracyFightWhitehall1977. Besides the polemics of the
left wing of the party, moderate or centrist Labour also lost faith in
the Civil Service. The socialist heralding of the bureaucracy as crucial
in realizing socialist goals was gradually abandoned, a mood stimulated
by the publication of popular books like Leslie Chapman's \emph{Your
Disobedient Servant}. Anthony Crosland, who in 1968 disagreed with the
New Left published his memoirs in which he accused the Civil Service of
obstruction during the first Wilson Government. Similar resentments
against an obstructive bureaucracy were voiced by Labour cabinet members
in the context of the IMF loan affair, during which the Treasury -
perhaps Labour's most important bureaucratic enemy - was allegedly
withholding information from Denis Healey, the Treasury chancellor.
Lastly, adding to the anti-bureaucratic conspiratory atmosphere in the
Callaghan years was the issue of secrecy and transparency. Controversies
over the government's White Paper on open government, but its refusal to
translate this in legislation, led to cross-party discontent with
"Whitehall men who prefer the cloak of secrecy" \#WhitehallMenWho1978.

With regard to the dynamics in the argumenative apparatus the Callaghan
years saw an intensification of specific arguments types, mostly the
ones pertaining to accountability and cost. However, arguments also
temporalised in a different way. The sense of continuous growth was
increasingly combined with the explicit expectation of renewal. To put
it in King's terms, the attention was no longer concentrated on the
rising waters, but on the sorcerer's return as well. The left wing
attacks on the "Whitehall mandarins" were framed as "a key
{[}struggle{]} for the next decade" \#DemocracyFightWhitehall1977. This
expectation of change in the 1980s intensified as the decade approached.
In March 1979, the Guardian published extensive articles with the
question "how the state will cope with the eighties", including telling
cartoons of a sweating Harold Wilson, overloaded with briefcases titled
'big goverment' \#HowWillState1979.

IMAGE: NGRAM A BUREAUCRACY

IMAGE: Cartoon

The discursive outcome of changes described in this paper - the
temporalisation, topicalisation and substantialisation of the
argumenative apparatus - are clearly present in a seemingly marginal
political debate during the last two years of the 1970s. In 1978 the
Conservatives introduced the term "quango" into the public debate. The
term, an abbreviation of "Quasi Non-Governmental Organisation", was
coined in the political science literature of the early 1970s. In the
late seventies, the Conservatives harnessed the concept to launch
another attack on the bureaucratic excesses of socialist govenrment.
Crucial in the accelerating use of the term was the book \emph{The
Quango Explosion}, published in 1978 by the Conservative Political
Centre - the educational wing of the party. In the same year, a Shadow
Cabinet review of the "quango's" was undertaken by Nigel Lawson, father
of Nigella Lawson and to-be Treasury Minister under Thatcher
\#coleQuangosDebate1970s2005. Taking a closer look at the first "quango
debates" - marked by the same ironic atmosphere as evoked by the term
itself: "are little quango's called quangaroos?" - it strikes how Labour
MPs immediately took over the term. The Conservative focal points -
inefficiency and elite appointments - were eagerly copied by MPs such as
Dennis Skinner and David Stoddart. Ioan Evans noted on the 19th of July
of the same year: "we should tackle the problem of the quango's. As a
party we must address ourselves to that problem" (Evans, Labour,
19-07-1979. Frequency plots for the lemma "quango" show the rapid
adoption of the term by Labour. This discursive copying, as well as the
extent to which Labour shared the Conservative diagnosis shows how at
the end of the 1970s, both parties shared a basic understanding of what
was wrong with government. Although their motives differed from time to
time, the many factions of the Labour Party generally agreed over the
inefficient, costly and unaccountable nature of government bureaucracy.

IMG: QUANGO FREQUENCY

\hypertarget{header-n77}{%
\subsection{Conclusion}\label{header-n77}}

In the period between 1957 and 1979 the argumentative apparatus around
the concept of "bureaucracy" changed substantively. Whereas in the 1950s
bureaucracy was predominantly used in a figurative sense, or was invoked
as a dangerous mirror image, the late 1960s and early 1970s presented a
different constellation of arguments that focused on bureaucracy as a
structural, urgent and administrative problem. Thinking about the state,
the civil service and its shape and character changed in a way that
enabled political actors in the late seventies to present an alternative
of small government. In this way, long-term conceptual and argumentative
change contributed to the, on first sight sudden, shift to New Right
policies.

\end{document}
