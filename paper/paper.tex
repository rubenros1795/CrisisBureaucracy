\documentclass{article}

\begin{document}

\section{Introduction}

- Introduction
"Today our image of government is more that of the sorcerer’s apprentice. The waters rise. The apprentice rushes about with his bucket. The waters rise even faster. And none of us knows when, or whether, the magician will come home".

In 1975 Anthony King, a political scientist provided Great Britain with what seemed at the time a perfect diagnosis of the age. Writing a year after the defeat of the Conservative Heath government, seen by many as a failure, King painted a grim picture of an unescapable high tide of governmental incapacity that even the newly installed Labour government was not able to curtail. In fact, the dynamic that King sought to describe  - that of electoral demands placed on an already overburdened government - was even expected to accelerate under Harold Wilson. Political promises would lead to increasing public expenditure and the continuing expansion of government, a process described one year earlier by Daniel Bell as the "revolution of the rising entitlements". In the years that followed, King's idea of ``overload" became the dominant frame in Conservative attacks on the Labour governments of Wilson and Callaghan. Margaret Thatcer argued in 1977 that "Government over-spending had overloaded the economy with a bureaucratic non-produtive sector weighing down wealth-creating industry. Smaller businesses suffered most". But the diagnosis was shared broadly. In an interview with the Guardian in 1975, Helmut Schmidt explained the British malaise by the firm believe in government investment and the growth of bureaucracy. Two years later, while reflecting on the future of Great-Britain, Prince Charles remarked in a 1977 radio interview that "we can expect to see an increasing bureaucracy". The Prince of Wales, referring to the Soviet Union, argued that "[i]f the experience of other countries is anything to go by, this will mean a gradual reduction in the freedom of choice and individual responsibility". 

The history of the 'New Right' turn is one of ideas. Current literature emphasizes the constructed nature of the various 'crises' that befell the 1970s Labour governments and the Keynesian policy paradigm. In his seminal article on the topic Colin Hay, for example, shows how the 'Winter of Discontent' - the winter of 1978/79 that saw widespread strikes and power outages - was both a political reality as a medial construct effectively exploited by Conservative strategists and ideologues (Hay, 1997). Hay's premise can be recognized in the field of constructivist institutionalism, that sees periods of economic crisis such as the late 1970s as times of 'Knightean uncertainty', when actors are not only uncertain about the solutions to crisis, but also about their interests in it.


- Problem statement / Question

- Data

- Approach


\section{Bureaucracy in Postwar British Political Thought}


\section{The Argumentative Apparatus of Bureaucracy}


\section{Arguments through Time}
- Conservative Years
- Wilson Government
- Heat Government
- Wilson Government
- Callaghan Government

\section{The Rising Waters}

\end{document}